\documentclass[a4paper,12pt]{article}

\usepackage[utf8]{inputenc}
\usepackage[german]{babel}
\usepackage{hyperref}
\usepackage[margin=3cm]{geometry}
\usepackage{graphicx}


\title{2D-Zeichnen im Browser}
\author{Johannes Reuter}
\date{\today}

\pdfinfo{%
  /Title    (2D-Zeichnen im Browser)
  /Author   (Johannes Reuter)
  /Creator  (Johannes Reuter)
  /Producer (Johannes Reuter)
}

\begin{document}
\maketitle
\tableofcontents
\section{Motivation}
\section{Anforderungen}
\section{HTML5-Canvas}
\subsection{Technologie}
\subsection{Frameworks}
\subsection{Beispiel}
\subsection{Performance}
\section{WebGL}
\subsection{Technologie}
\subsection{Frameworks}
\subsection{Beispiel}
\subsection{Performance}
\section{DOM}
\subsection{Technologie}
\subsection{Frameworks}
\subsection{Beispiel}
\subsection{Performance}
\section{Fazit}
\section{Ausblick}
\end{document}
